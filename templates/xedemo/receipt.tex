<?lsmb FILTER latex { format="$FORMAT(xelatex)" };
       INCLUDE "preamble-xelatex" -?>

\begin{document}

\pagestyle{myheadings}
\thispagestyle{empty}

\parbox[t]{12cm}{
  <?lsmb company ?>

  <?lsmb address ?>}
\hfill
\parbox[t]{6cm}{\hfill <?lsmb source ?>}

\vspace*{0.6cm}

<?lsmb text_amount ?> \dotfill <?lsmb decimal ?>/100 \makebox[0.5cm]{\hfill}

\vspace{0.5cm}

\hfill <?lsmb datepaid ?> \makebox[2cm]{\hfill} <?lsmb amount ?>

% different date format for datepaid
% <?lsmb DD ?><?lsmb MM ?><?lsmb YYYY ?>

\vspace{0.5cm}

<?lsmb name ?>

<?lsmb address1 ?>

<?lsmb address2 ?>

<?lsmb city ?>
<?lsmb IF state ?>
\hspace{-0.1cm}, <?lsmb state ?>
<?lsmb END ?>
<?lsmb zipcode ?>

<?lsmb country ?>

\vspace{1.8cm}

<?lsmb memo ?>

\vspace{0.8cm}

<?lsmb company ?>

\vspace{0.5cm}

<?lsmb name ?> \hfill <?lsmb datepaid ?> \hfill <?lsmb source ?>

\vspace{0.5cm}
\begin{tabularx}{\textwidth}{lXrr@{}}
\textbf{<?lsmb text('Invoice No.') ?>} & \textbf{<?lsmb text('Invoice Date') ?>}
  & \textbf{<?lsmb text('Due') ?>} & \textbf{<?lsmb text('Applied') ?>} \\
<?lsmb FOREACH invnumber ?>
<?lsmb lc = loop.count - 1 ?>
<?lsmb invnumber.${lc} ?> & <?lsmb invdate.${lc} ?> \dotfill
  & <?lsmb due.${lc} ?> & <?lsmb paid.${lc} ?> \\
<?lsmb END ?>
\end{tabularx}

\vspace{1cm}

<?lsmb memo ?>

\vfill

\end{document}
<?lsmb END -?>
