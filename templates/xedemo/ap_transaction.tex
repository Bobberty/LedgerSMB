<?lsmb FILTER latex { format="$FORMAT(xelatex)" };
       INCLUDE "preamble-xelatex" -?>

\begin{document}

\pagestyle{myheadings}
\thispagestyle{empty}

<?lsmb INCLUDE letterhead ?>

\centerline{\MakeUppercase{\textbf{<?lsmb text('AP Transaction') ?>}}}

\vspace*{0.5cm}

\parbox[t]{.5\textwidth}{
<?lsmb name ?>

<?lsmb address1 ?>

<?lsmb address2 ?>

<?lsmb city ?>
<?lsmb IF state ?>
\hspace{-0.1cm}, <?lsmb state ?>
<?lsmb END ?> <?lsmb zipcode ?>

<?lsmb country ?>

\vspace{0.3cm}

<?lsmb IF contact ?>
<?lsmb contact ?>
\vspace{0.2cm}
<?lsmb END ?>

<?lsmb IF vendorphone ?>
Tel: <?lsmb vendorphone ?>
<?lsmb END ?>

<?lsmb IF vendorfax ?>
Fax: <?lsmb vendorfax ?>
<?lsmb END ?>

<?lsmb email ?>

<?lsmb IF vendortaxnumber ?>
Tax Number: <?lsmb vendortaxnumber ?>
<?lsmb END ?>
}
\hfill
\begin{tabular}[t]{ll}
  \textbf{<?lsmb text('Invoice #') ?>} & <?lsmb invnumber ?> \\
  \textbf{<?lsmb text('Date') ?>} & <?lsmb invdate ?> \\
  \textbf{<?lsmb text('Due') ?>} & <?lsmb duedate ?> \\
  <?lsmb IF ponumber ?>
    \textbf{<?lsmb text('PO #') ?>} & <?lsmb ponumber ?> \\
  <?lsmb END ?>
  <?lsmb IF ordnumber ?>
    \textbf{<?lsmb text('Order #') ?>} & <?lsmb ordnumber ?> \\
  <?lsmb END ?>
  \textbf{<?lsmb text('Employee') ?>} & <?lsmb employee ?> \\
\end{tabular}

\vspace{1cm}

\begin{tabularx}{\textwidth}[t]{@{}llrX@{\hspace{1cm}}l@{}}
<?lsmb FOREACH amount ?>
<?lsmb lc = loop.count - 1 ?>
  <?lsmb accno.${lc} ?> &
  <?lsmb account.${lc} ?> &
  <?lsmb amount.${lc} ?> &
  <?lsmb description.${lc} ?> &
  <?lsmb projectnumber.${lc} ?> \\
<?lsmb END ?>

  \multicolumn{2}{r}{\textbf{Subtotal}} & <?lsmb subtotal ?> & \\
<?lsmb FOREACH tax ?>
<?lsmb lc = loop.count - 1 ?>
  \multicolumn{2}{r}{\textbf{<?lsmb taxdescription.${lc} ?> @ <?lsmb taxrate.${lc} ?> \%}} & <?lsmb tax.${lc} ?> & \\
<?lsmb END ?>

  \multicolumn{2}{r}{\textbf{Total}} & <?lsmb invtotal ?> & \\
  
\end{tabularx}

\vspace{0.3cm}

<?lsmb text_amount ?> ***** <?lsmb decimal ?>/100 <?lsmb currency ?>

<?lsmb IF notes ?>
\vspace{0.3cm}
<?lsmb FOREACH P IN notes.split('\n\n+') ?>
<?lsmb P ?>\medskip

<?lsmb END ?>
<?lsmb END ?>

\vspace{0.3cm}

<?lsmb IF paid_1 ?>
\begin{tabular}{@{}llllr@{}}
  \multicolumn{5}{c}{\textbf{<?lsmb text('Payments') ?>}} \\
  \hline
  \textbf{<?lsmb text('Date') ?>} & & \textbf{<?lsmb text('Source') ?>} & \textbf{<?lsmb text('Memo') ?>} & \textbf{<?lsmb text('Amount') ?>} \\
<?lsmb END ?>
<?lsmb FOREACH payment ?>
<?lsmb lc = loop.count - 1 ?>
  <?lsmb paymentdate.${lc} ?> & <?lsmb paymentaccount.${lc} ?> & <?lsmb paymentsource.${lc} ?> & <?lsmb paymentmemo.${lc} ?> & <?lsmb payment.${lc} ?> \\
<?lsmb END ?>
<?lsmb IF paid_1 ?>
\end{tabular}
<?lsmb END ?>

\end{document}
<?lsmb END -?>
