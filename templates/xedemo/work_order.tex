<?lsmb FILTER latex { format="$FORMAT(xelatex)" } -?>
\documentclass{scrartcl}
\usepackage{xltxtra}
\usepackage{tabularx}
\usepackage{fontspec}
\usepackage{longtable}
\setmainfont{LiberationSans-Regular.ttf}
\usepackage[letterpaper,top=2cm,bottom=1.5cm,left=1.1cm,right=1.5cm]{geometry}
\usepackage{graphicx}
\setlength{\parindent}{0pt}

\begin{document}

\pagestyle{myheadings}
\thispagestyle{empty}

<?lsmb INCLUDE letterhead ?>


% Break old pagebreak directive
%<?xlsmb pagebreak 65 27 48 ?>
%\end{tabularx}
%
%\newpage
%
%\markboth{<?xlsmb company ?>\hfill <?xlsmb ordnumber ?>}{<?xlsmb company ?>\hfill <?xlsmb ordnumber ?>}
%
%\begin{tabularx}{\textwidth}{@{}rlXrll@{}}
%  \textbf{Item} & \textbf{Number} & \textbf{Description} & \textbf{Qt'y} &
%  & \textbf{Serial Number} \\
%<?xlsmb end pagebreak ?>


\vspace*{0.5cm}

\parbox[t]{.5\textwidth}{
\textbf{To}
\vspace{0.3cm}

<?lsmb name ?>

<?lsmb address1 ?>

<?lsmb address2 ?>

<?lsmb city ?>
<?lsmb IF state ?>
\hspace{-0.1cm}, <?lsmb state ?>
<?lsmb END ?>
<?lsmb zipcode ?>

<?lsmb country ?>

\vspace{0.3cm}

<?lsmb IF contact ?>
<?lsmb contact ?>
\vspace{0.2cm}
<?lsmb END ?>

<?lsmb IF customerphone ?>
<?lsmb text('Tel: [_1]', customerphone) ?>
<?lsmb END ?>

<?lsmb IF customerfax ?>
<?lsmb text('Fax: [_1]', customerfax) ?>
<?lsmb END ?>

<?lsmb email ?>
}
\parbox[t]{.5\textwidth}{
\textbf{Ship To}
\vspace{0.3cm}

<?lsmb shiptoname ?>

<?lsmb shiptoaddress1 ?>

<?lsmb shiptoaddress2 ?>

<?lsmb shiptocity ?>
<?lsmb IF shiptostate ?>
\hspace{-0.1cm}, <?lsmb shiptostate ?>
<?lsmb END ?>
<?lsmb shiptozipcode ?>

<?lsmb shiptocountry ?>

\vspace{0.3cm}

<?lsmb IF shiptocontact ?>
<?lsmb shiptocontact ?>
\vspace{0.2cm}
<?lsmb END ?>

<?lsmb IF shiptophone ?>
<?lsmb text('Tel: [_1]', shiptophone) ?>
<?lsmb END ?>

<?lsmb IF shiptofax ?>
<?lsmb text('Fax: [_1]', shiptofax) ?>
<?lsmb END ?>

<?lsmb shiptoemail ?>
}
\hfill

\vspace{1cm}

\textbf{\MakeUppercase{<?lsmb text('Work Order') ?>}}
\hfill

\vspace{1cm}

\begin{tabularx}{\textwidth}{*{6}{|X}|} \hline
  \textbf{Order \#} & \textbf{Order Date} & \textbf{Required by} & \textbf{Salesperson} & \textbf{Shipping Point} & \textbf{Ship Via} \\ [0.5em]
  \hline
  <?lsmb ordnumber ?> & <?lsmb orddate ?> & <?lsmb reqdate ?> & <?lsmb employee ?> & <?lsmb shippingpoint ?> & <?lsmb shipvia ?> \\
  \hline
\end{tabularx}

\vspace{1cm}

\begin{longtable}{@{\extracolsep{\fill}}rllrll@{}}
  \textbf{<?lsmb text('Item') ?>} & \textbf{<?lsmb text('Number') ?>}
  & \textbf{<?lsmb text('Description') ?>} & \textbf{<?lsmb text('Qty') ?>} &
  & \textbf{<?lsmb text('Serial Number') ?>} \\
<?lsmb FOREACH number ?>
<?lsmb lc = loop.count - 1 ?>
  <?lsmb runningnumber.${lc} ?> &
  <?lsmb number.${lc} ?> &
  <?lsmb item_description.${lc} ?> &
  <?lsmb qty.${lc} ?> &
  <?lsmb unit.${lc} ?> &
  <?lsmb serialnumber.${lc} ?> \\
<?lsmb END ?>
\end{longtable}


\parbox{\textwidth}{
\rule{\textwidth}{2pt}

\vspace{12pt}

<?lsmb FOREACH P IN notes.split('\n\n') ?>
<?lsmb P ?>\medskip

<?lsmb END ?>
}

\vfill

\end{document}
<?lsmb END -?>
